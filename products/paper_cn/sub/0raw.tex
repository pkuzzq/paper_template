\subsection{基本框架}

Present bias

habitat formation

utilitity function 如何改变的

• 拖延和提前准备

必须行动假设。一个人必须在 $T$个周期内,一次性完成一项任务或活动。若过去从未进行过该任务,那么下一个周期的个人必须决定是否进行或推迟活动。然而,一旦完成该任务,个人在未来的任何周期中都没有进一步的行动需要。

• 行动的成本和收益

在 $t$ 时期,完成任务的成本和收益分别为 $\nu_t \geq 0$ 和 $c_t \geq 0$

假设:在 $T$ 周期内有 $T$ 个独立的自我,在 $t$ 时期的自我不能约束未来时期 $\tau$ 的自我行动,


\subsubsection{多重自我与准双曲线模型}

当引入时间维度后,心理学中若个时期中,同一个个体划分成为了不同的自我(distinct self),因此随着时间推移,存在同一个体就存在多重自我(muliple selvs),


在决定未来自我是否执行任务,在任意的未来时期都有相同的信念偏好$(\hat\beta,1)$。依赖于$\beta$和$\hat\beta$的关系,信念将个体分成三种类型,时间一致者(time consistents)、成熟老练者(sophsticates)和天真者(naifs)。
多重自我研究中的一个重要问题是,当前自我对未来自我可能发生现时偏误时的意识程度。完全的成熟老练者,能够充分意识到未来的自己会有自我控制问题;然而天真者则完全不会意识到自我控制问题的发生。

决策重点是:不同类型个体会在什么时间开始执行任务?不同的成本与收益是否对执行的时间有影响?


• 偏好假设

参照Laibson(1997)和O'Donoghue和Rabin(1999b)所假设的,每个$t$时段的自我都具有准双曲线型偏好(quasi-hyperbolic preference)$(\beta , \delta)$,该偏好可能具有时间不一致性,表现为现时偏误。

$$U_t\left(c_t, c_{t+1}, \ldots, c_T\right)=u\left(c_t\right)+\beta \sum_{\tau=t+1}^{\tau=T} \delta^\tau u\left(c_\tau\right), 0<\beta<1$$

其中 $\delta \in(0,1]$ 和 $\beta \in(0,1]$ 。其中,参数$\delta$ 是指数型贴现效用函数(exponential discounted utility,EDU)中长期、时间一致的折现因子;参数$\beta$ 反映了每个自我对现时满足感的偏误。若$\beta = 1$ 代表跨期偏好是时间一致的,若$\beta < 1$ 则意味着跨期偏好是存在现时偏误,因为对$t$时期的瞬时效用所赋予的相对权重比未来效用所赋予的权重更大。依据必须行动假设,在任意$t \in \Gamma=\{0,1,2, \ldots, T\}$时期,个体必须形成关于所有未来的自我的行为信念,即对未来$k>t$时期的自我意识程度(degree of  self-awareness):

$$U_k\left(c_k, \ldots, c_T\right)=u\left(c_k\right)+\widehat{\beta} \sum_{\tau=k+1}^{\tau=T} \delta^\tau u\left(c_\tau\right), t<k \leq T-1$$









