%!TEX program = xelatex
% 完整编译: xelatex -> biber/bibtex -> xelatex -> xelatex
\documentclass[lang=cn,22pt,a4paper]{elegantpaper}

\title{作论文模板}
\author{Zhengqing ZHOU \thanks{北京大学} \\ Peking University \and Zhengqing ZHOU \\ PA Technology}
\institute{\href{https://pe.pku.edu.cn/}{体育教研部}}

\version{0.10}
\date{\zhtoday}

\usepackage{setspace}
\usepackage{amsmath}
\usepackage{amssymb}
\usepackage{graphicx}
\usepackage[utf8]{inputenc}
\setlength{\parskip}{0.1in}%段间距
\usepackage[autostyle]{csquotes}

% Packages for automated tables
\usepackage{tabularx}
\usepackage{standalone}
\usepackage{pdflscape}
\usepackage{iftex, fancyhdr, hyperref, enumitem, fancyvrb, hologo, multirow, booktabs, bigstrut, tabularx, tocloft}
\usepackage{lipsum}

\usepackage{changes} %批注
\usepackage{bm}
\hypersetup{colorlinks = true, allcolors = blue}
\pagestyle{fancy}\fancyhf{}\cfoot{\thepage}
\renewcommand*{\headrulewidth}{0pt}
\ctexset{linestretch = {\maxdimen}}
\setlength{\hfuzz}{1.5pt}
\setlist{nolistsep}

% 本文档命令
\usepackage{array}
\newcommand{\ccr}[1]{\makecell{{\color{#1}\rule{1cm}{1cm}}}}
\addbibresource[location=local]{reference.bib} % 参考文献,不要删除

\begin{document}

% Title Page
\begin{titlepage}
\clearpage\maketitle
\thispagestyle{empty}

\begin{abstract}
本文为的说明文档
 \par
【Background:】
\par
【Purpose:】
\par
【Methods/Design:】 
\par
【Results:】 
\par
【Conclusions:】

\keywords{Elegant\LaTeX{},工作论文,模板}
\end{abstract}

\end{titlepage}
    \thispagestyle{empty}
	\newpage
    \phantomsection\addcontentsline{toc}{section}{摘要}\tolerance=500 %将摘要放进目录
    \tableofcontents
    \setcounter{page}{1}
	\newpage

% 理论、变量、处理手法、基本结论
	\section*{研究笔记} \label{sec:text}
	\subsection{基本框架}

Present bias

habitat formation

utilitity function 如何改变的

• 拖延和提前准备

必须行动假设。一个人必须在 $T$个周期内,一次性完成一项任务或活动。若过去从未进行过该任务,那么下一个周期的个人必须决定是否进行或推迟活动。然而,一旦完成该任务,个人在未来的任何周期中都没有进一步的行动需要。

• 行动的成本和收益

在 $t$ 时期,完成任务的成本和收益分别为 $\nu_t \geq 0$ 和 $c_t \geq 0$

假设:在 $T$ 周期内有 $T$ 个独立的自我,在 $t$ 时期的自我不能约束未来时期 $\tau$ 的自我行动,


\subsubsection{多重自我与准双曲线模型}

当引入时间维度后,心理学中若个时期中,同一个个体划分成为了不同的自我(distinct self),因此随着时间推移,存在同一个体就存在多重自我(muliple selvs),


在决定未来自我是否执行任务,在任意的未来时期都有相同的信念偏好$(\hat\beta,1)$。依赖于$\beta$和$\hat\beta$的关系,信念将个体分成三种类型,时间一致者(time consistents)、成熟老练者(sophsticates)和天真者(naifs)。
多重自我研究中的一个重要问题是,当前自我对未来自我可能发生现时偏误时的意识程度。完全的成熟老练者,能够充分意识到未来的自己会有自我控制问题;然而天真者则完全不会意识到自我控制问题的发生。

决策重点是:不同类型个体会在什么时间开始执行任务?不同的成本与收益是否对执行的时间有影响?


• 偏好假设

参照Laibson(1997)和O'Donoghue和Rabin(1999b)所假设的,每个$t$时段的自我都具有准双曲线型偏好(quasi-hyperbolic preference)$(\beta , \delta)$,该偏好可能具有时间不一致性,表现为现时偏误。

$$U_t\left(c_t, c_{t+1}, \ldots, c_T\right)=u\left(c_t\right)+\beta \sum_{\tau=t+1}^{\tau=T} \delta^\tau u\left(c_\tau\right), 0<\beta<1$$

其中 $\delta \in(0,1]$ 和 $\beta \in(0,1]$ 。其中,参数$\delta$ 是指数型贴现效用函数(exponential discounted utility,EDU)中长期、时间一致的折现因子;参数$\beta$ 反映了每个自我对现时满足感的偏误。若$\beta = 1$ 代表跨期偏好是时间一致的,若$\beta < 1$ 则意味着跨期偏好是存在现时偏误,因为对$t$时期的瞬时效用所赋予的相对权重比未来效用所赋予的权重更大。依据必须行动假设,在任意$t \in \Gamma=\{0,1,2, \ldots, T\}$时期,个体必须形成关于所有未来的自我的行为信念,即对未来$k>t$时期的自我意识程度(degree of  self-awareness):

$$U_k\left(c_k, \ldots, c_T\right)=u\left(c_k\right)+\widehat{\beta} \sum_{\tau=k+1}^{\tau=T} \delta^\tau u\left(c_\tau\right), t<k \leq T-1$$










	% Research Questions 研究问题提炼
	\newpage
	\section{研究问题} \label{sec:research questions}
	\subsection{SI and PA}

% Introduction引言
	\newpage
	\section{引言} \label{sec:introduction}
	\input{./sub/2introduction.tex}
% Section文献综述
	\newpage
	\section{文献回顾} \label{sec:review}
	\input{./sub/3review.tex}
% Section文献综述
	\newpage
	\section{研究设计} \label{sec:design}
	\input{./sub/4data_methods.tex}
% Section方法
    \newpage
    \section{数据来源与数据描述}\label{sec:data}
	\input{./sub/5sample.tex}
% Section实证结果
    \newpage
    \section{研究结果与分析}\label{sec:results}
	\input{./sub/6results.tex}
% Conclusion
	\newpage
	\section{结论} \label{sec:conclusion}
	\input{./sub/7conclusion.tex}
% References
	\newpage
%	\bibliographystyle{apa}
%	\bibliography{}
	\nocite{*}
	\printbibliography[heading=bibintoc, title=\ebibname]
% Appendix
	\newpage
	\appendix
	\section{附录} \label{sec:appendix}
	
% FIGURES
\subsection{Figures}

\begin{figure}[H]
\caption{}
\centering
%\includegraphics[scale=1]{../../output/}
\label{fig:}
\end{figure}


% TABLES
\newpage
\subsection{Tables}

\begin{table}[H]
\caption{}
\centering
\begin{threeparttable}
%\input{../../output/}
\end{threeparttable}
\label{tab:}
\end{table}

	\addappheadtotoc

\end{document}