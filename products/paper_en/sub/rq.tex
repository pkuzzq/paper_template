\subsection{SI and PA}
Q1. does Social Integration affect PA?

A1. possitive 

Hypothesis 1: All other variables being equal, the higher the Social Integration of children, the more time they will spending time on physical activities. And there is heterogeneity in different groups.


Hypothesis 2: All other variables being equal, diversity in the classroom has an inverted "U" shaped relationship with with the behaviors among adolescents' cross-group friends.

\subsection{Class Divsity and PA}

Q2. does Social Integration affect PA?

A1. ? 

Hypothesis 3: All other variables being equal, The greater the density of cross-group interactions among the groups to which they belong in the class, the greater the number of adolescents' cross-group friends.

Hypothesis 4: The higher the density of homogeneous external group interactions, the lower the number of adolescents' cross-group friends, all other variables being consistent.

\subsection{Migrate and local differences in Class Divsity influences}

Hypothesis 5: There will be significant differences in the effects of group composition of the class, cross-group interaction density of the group to which they belong, and homogeneous interaction density of the external group on the number of cross-group friends of mobile and local children.

通过已有研究可以发现,首先,目前大多数研究对教育获得因果机制的探究中,只有少数对比了不同家庭维度对孩子教育获得成果的影响。其次,国内外的研究更加侧重家庭静态因素对子女教育获得的影响,较少关注家长参与、情感投入等动态因素,而对比这两条不同路径对子女认知能力的影响的研究则更少。在仅有的比较不同类型的教育投资对认知能力影响的研究中,对家庭经济条件的测量则过于简单,未将家庭多元的教育投入纳入考虑,如课外补习花费、校内学习花费等具体经济变量。最后,国内关于学生认知能力的研究大都使用了中国教育追踪调查数据(CEPS)2013-2014年基线数据,而该数据在许多关于教育投入的变量上存在大量的缺失情况,以往的研究中也很少有学者关照到这些数据缺失的情况并对其进行处理。因此,本文以一个动态的视角来比较不同类型的家庭教育投入对孩子认知能力的影响,综合运用不同维度的理论视角作为指导,同时也采取更细化的指标及完整的数据进行研究,以期能更加科学和严谨。