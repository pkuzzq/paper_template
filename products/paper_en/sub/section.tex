The level of family environment and nurturing during childhood determines the skill base of an individual throughout life and is a key factor in human development (Heckman and Mosso,2014). Playing a key role in explaining the intergenerational mobility of economic status, children growing up in favorable and unfavorable environments have human capital gaps in human capital emerge well before school entry; and traditional education policies cannot fully compensate for the injuries caused by underinvestment by parents in disadvantaged families Traditional education policies cannot fully compensate for the damage caused by underinvestment by parents in disadvantaged families. Children growing up in different family backgrounds face what is known as the "fork in the road of destiny," or inequality of opportunity in human capital investment.



