
    • Topic and Variable 
    
    • Which One  
     
    • follow Bobby W. Chung \& Jian Zou

    Step 1:
        
    We utilize random assignment of students into classrooms in China middle schools to study the mechanisms behind the spillover of peer parental education on student achievement. 
    
    Analyzing the China Education Panel Survey, we find a    causal relationship between X and Y. 
    
    In addition to the conventional peer effect and teacher response channel, we identify mother adjustment of parenting style as another important mediating factor. 
    
    find W as mediating factor.
    
    We provide suggestive evidence about the existence of the mother’s network, which facilitates the change in parenting style. 
    
    We also find that the spillover of peer maternal education on non-repeaters and non-migrant students is stronger, primarily driven by higher parental investment in time.
    
we follow the method by Gong,Lu, and Song(2018), which is classroom assignment randomization. Our selected sample is almost identical to that of Gong, Lu, and Song (2018).

Our Y is PA , measured by the question:

In the raw data, the scores of the three subjects have been standardized with a mean of 70 and a standard deviation of 10. Then we generate the standardized PA timing with a mean of 0 and a standard deviation of 1.

Throughout this paper, groups are defined as migrant and local students. in each one, there are 11111 students on average, and in each class per group with a minimum of 21 and a maximum of 81.

The key variable of interest are three: one, Social Integration (assi); two, class diversity with some characters (made of an index); three, check the character of parents.