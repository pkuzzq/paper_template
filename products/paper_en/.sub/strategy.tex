\subsection{Class Assignment and Regression Sample}

为验证样本“随机分班”的可靠性,我们分别从学生和教师特征进行检验。 表 3-2 从全样本和分年级样本检验学生个体先决特征变量与班级其他同学特征 的相关性。借鉴 Sacerdote et al.(2001)、Guryan et al.(2009) 和 Eisenberg(2014) 的 随机检验方法,如果学校内部随机分班的假设成立,在控制学校年级固定效应 后,个体的先决特征变量和同伴的平均特征无关,即个体的先决特征变量对同 伴该变量的均值做回归,其回归系数应该不显著。Guryan et al.(2009) 指出在样本有限的情况下,该检验存在固有偏差,因为个体不可能是自己的同伴,即使 在随机分配的样本中,个体的先决特征变量和同伴的平均特征也出现显著的负 相关。比如一个高能力个体的同伴能力要比低能力个体的同伴能力更差。而如 果回归中出现显著的正相关,则表明学生在班级之间存在相似的学生被分配到 相同班级的现象,随机分班的假设不成立。

表 3-2 中列出了分别以个体每个背景特征为因变量,对同伴相应的先决特 征变量的均值进行回归的系数。结果显示个别变量(全样本中的学生性别、母 亲教育水平和父亲职业、九年级中的农业户口)存在系数显著且为负的情况, 大部分变量的系数不显著,没有出现正向显著相关的情况。根据 Guryan et al.(2009) 指出的在样本有限的情况下,这些负相关属于随机检验中存在的固有偏 差。因此总体上看,并没有发现学生与其同伴特征有显著的正向选择问题。 为检验教师是否随机被分配到每个班级,表 3-3 汇报了控制学校年级固定 效应后,班主任的每个先决特征(班主任性别、班主任教学经验、班主任学历 和班主任年龄)对班级学生和其父母的每个背景特征的均值分别进行回归的系 数。结果显示回归系数都不显著,表明在随机分班的学校样本中,教师特征与 班级学生和其父母特征之间并不存在显著的相关性。


因果推断的有效性依赖于随机分班的可靠性。鉴于学生家庭背景特征是影 响学生行为表现的重要因素,我们在表 4-3 中分别用因变量对除了班级高能力 学生比例以外的其他学生个体、家庭背景特征和班主任特征先决变量进行回归, 同时控制学校年级固定效应。回归结果表明学生性别、父母的教育水平、家庭 经济状况等变量显著影响学生的认知和非认知能力。在确认了影响学生行为表 现的先决变量之后,为了估计随机分班的可靠性,我们借鉴 Xu et al.(2020) 的随 机分班检验方法,用班级高能力学生比例对普通学生的个体和家庭背景特征先 决变量进行回归来检验随机分班的可靠性。如果随机分班被严格执行了,那么 学生个体和家庭背景特征先决变量和高能力学生的比例不相关,其回归系数应 该不显著。该结果呈现在表 4-3 的第(5)列,结果显示,除了学生性别和班主 任学历外,其他先决变量的系数都很小并且都不显著。因此尚未发现学生被分 配到高能力学生比例较高班级的可能性。 另一种验证样本“随机分班”可靠性的方法是从学生和教师两个方面进 行随机分班检验。我们借鉴同样使用 CEPS 随机分班样本数据的 Gong et al. (2018,2021) 中平衡性检验(balancing tests)的方法,如果随机分班假设成立, 班级中高能力学生的比例与普通学生个体背景特征和教师特征的先决变量(如 户口类型、独生子女、父母教育水平、家庭经济状况、任课老师的性别、年龄、 学历、教龄等等)无关。

因此,表 4-4 中列出了基于学校年级固定效应模型分别以普通学生的个体 背景特征和教师特征为因变量对班级高能力学生比例进行回归的系数。从表 4-4 中第一列的检验学生是否随机被分配到每个班级的回归结果看,在相同学校相 同年级中,除了学生性别、上过幼儿园和小学留过级变量以外,高能力学生的 比例与普通学生个体和家庭背景的大部分特征都没有显著的关系。

个别变量在组间有显著性差异在研究同伴效应文献的随机分班平衡性检验中并不少见,如:Gong et al (2018, 2021)、 Wang et al (2018)、 Lavy et al (2012)、 Carrell et al.(2013) 和 Carrell et al.(2010) 等等。从表4-4中第二列的检验教师是否随机被分配到每个班级的回归结果看,回归系数几乎都不显著,表明在随机分班的学校样本中,教师特征与班级学生特征之间并不存在显著的相关性,我们可以排除学校分配更好的老师给高能力学生比例更高班级的可能性。因此, 我们可以基本证实随机分班的可靠性。

\subsection{Methodology}

根据前述研究假设,本文仍采用传统线性均值模型,并扩展了Ryan(2017)的方法四,将基准模型设定为方程\eqref{eq:basic.reg}来验证班级中的同伴效应对初中生体育锻炼行为的影响。

\begin{equation}
PA_{i,c,s} =\beta_0 + \beta_1 * {ParSty}_{i,c,s}+\beta_2 * {SoG}_{-ics} + \boldsymbol{\gamma} \boldsymbol{X}_{i,c,s} +\boldsymbol{\gamma}\boldsymbol{P}_{i,c,s}+ \omega_g+\lambda_s+\delta_{gs}+\varepsilon_{i,c,s}
\label{eq:basic.reg}
\end{equation}

where Y.ih refers to the test score of student i in class i in a school-grade k. Mother Ed.n元, is the leave-me-out average college attainment of mothers in a classroom. Xi,k is a vector of control variables listed in Table 1.8 We isolate the effect of teacher characteristics (TY,k, On students by including head teachers’ age, gender, marriage status, whether having a college degree, teaching experience in years, whether graduated from a normal college, and whether having a teacher certificate (Gong, Lu and Song, 2018).

We include school-by-grade fixed effect (x) to control for school and residential sorting.

With the fixed effect, the variation of Mother Edu-ii,h then entirely comes from within-grade classroom composition, which is the level the randomization happens. Our randomization setting is rare and unique, especially with a national sample of students. An analogous dataset in the US is the National Longitudinal Study of Adolescent to Adult Health (Addhealth), which contains actual friendship detail for a national sample of US students and is widely used in the peer effect iterature. Because of the concern over individual sorting into peer groups, researchers require additional assumptions to identify the spillover of peer parental education (Bifulco, Fletcher and Ross, 2011; Bifulco et al., 2014; Olivetti, Patacchini and Zenou, 2018; Chung, 2020).

\subsection{Balance Test}

这部分包含:平衡性检验

因此,本研究借助 CEPS 数据特有的分配规则,将样本限制在那些随机分 配学生的学校,以此来解决潜在的内生同伴选择问题。研究的关键识别假设是, 学生是被随机或者平均地分配到班级中去。为此,首先按照第二节中提供的分配 规则信息,识别出本研究的样本;然后进一步利用统计性检验来验证这一随机分 配假设的有效性。 具体而言,这里采用两个统计性检验。第一个统计性检验一般被称为平衡性 检验(Balancing test)38。在具体的回归分析中,将第二节中提到的个体层面的 事前控制变量对班级中受过学前教育的学生占比进行回归;以及将班级层面的控 制变量对个人是否接受学前教育进行回归。如果学生在一个学校里被随机分配到 任意班级,那么班级中接受学前教育的学生比例变量与那些观察到的、可能破坏 随机分配规则的个体特征之间不应该存在统计相关性。同理,如果学生是被随机 分配的,那么个人是否接受学前教育这一事前变量,也不会与班级层面的变量有 关联。举例而言,假设接受过学前教育的孩子的父母,他们更重视人力资本投资, 因此希望孩子可以进入某个名师的课堂,或者说会选择让孩子进入有某种家庭背 景的学生聚集的班级,那么就可能看到个人是否接受学前教育这个变量就会与那 些班级层面的变量显著相关。所以,如果上述回归结果中存在较多变量的估计系 数在统计上显著,就表明此样本数据违反了随机分配假设,而估计结果也将存在 偏误。

Before we estimate the spillover of peers’ maternal education, we verify the peer group assignment is as good as random. Because our identification relies on within-grade variations of classroom composition, we need to make sure there are ample variations of the treatment variable.

图1a显示了X变量在各班级中的原始分布,其中x轴代表班级标识符,y轴代表班级中X比变量的比例。我们在图1b中进一步将这些数值组织成柱状图。如图所示,班级中拥有X变量从0到63.64个百分点不等。在图2a中,我们显示了在同一学校年级的条件下,不同班级间X变量的相关性。每个点代表一个学校年级,X轴是班级1的X的变量比例,Y轴是另一个班级2的X变量。该图的想法是看看两个班级是否相同的数值,在这种情况下,点将正好落在45度线上。如图所示,许多点都是分散的,这意味着即使控制了按年级的固定效应,我们的处理变量也有很大的变化。在同一硬币的另一面,图2b描绘了大学毕业的母亲比例的年级内班级间差异的柱状图。差异范围从0到28.30个百分点,超过60%的班级对没有相同的值。这再次表明,我们的处理变量有足够的变化来进行识别。!!

作为本文最重要的识别手段,下文将继续对上述52所样本学校的“随机分班”特征进行检验。具体来说,本文根据同校内的两个班级设定虚拟变量,进而考察其与学生在7年级时的心理健康状况及其他个人、家庭、父母、班级和学校特征之间的关系,回归结果见表2。可见,几乎所有基线变量都不与班级虚拟变量存在显著关系①,这在一定程度上支持了样本校的分班随机性。

在对样本校的随机分班特征进行检验后,另一个需要注意的问题是,学校在各班级之间的资源分配是否平衡。例如,为完成某些教学目标,学校可能在随机分班后对某班级投入更多的教育资源。因此,参考王海宁等的研究(Wangetal.,2018),本文将继续考查学生及其家庭可观测特征的班级均值(例如学生年龄、性别、民族、出生次序、兄弟姐妹数量、户口、家庭经济条件及父母受教育年限)与班级和教师特征(例如班级规模、教师个人特征及行为特征)之间的关系,回归结果见表3。根据检验结果,在加入学校固定效应的条件下,几乎所有系数都不显著(仅存在四个例外,但是估计值均较小且可能来源于样本误差)。可见,对于随机分班的样本校,其在校内各班级之间的资源分配也没有表现出显著倾向。

我们还通过观察LBC在学校和年级中的比例变化是否与几个预定的学生特征的变化有关,来研究识别策略的有效性。如果这个分配过程是真正的随机的,那么学生在这些预定的特征上应该是相似的[29]。我们分别进行回归,其中因变量是学生的预定特征,而班级中LBC的份额是自变量,加入年级和学校的固定效应和学校特定年级的趋势效应。表3中的证据支持上述识别假设的有效性。也就是说,学生的预定特征在具有不同LBCs学生份额的班级中是平衡的。

我们加入了按年级的固定效应(ox)来控制学校和居住地的排序。有了固定效应,母亲Edu-ij.k的变化就完全来自年级内的教室构成,这就是随机化发生的水平。我们的随机化设置是罕见和独特的,特别是在全国性的学生样本中。在美国,一个类似的数据集是全国青少年到成人健康纵向研究(Addhealth),它包含了美国学生的全国性样本的实际友谊细节,并被广泛用于同伴效应文献。由于对个人分类进入同伴群体的担忧,研究人员需要额外的假设来确定同伴父母教育的溢出(Bifulco,Fletcher和Ross,2011;Bifulco等人,2014;Olivetti,Patacchini和Zenou,2018;Chung,2020)。

As an additional check, we performed Monte Carlo simulations for the elementary, middle, and high school samples to verify that the observed within school variation in the proportion of female students was consistent with a random process. For each school, we randomly generated the gender of the students in each cohort and computed the within school standard deviation of the proportion female.17 We repeated this process 1,000 times to obtain an empirical 90 percent confidence interval for the standard deviation for each school.


