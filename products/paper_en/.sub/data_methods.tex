\subsection{Data}

According to Manski (1993), the use of a randomly assigned sample is necessary to remove the group factor of interest from the correlation effect, so this paper selects a "randomly assigned" sample of schools in which new students are randomly assigned at entry and are not reassigned thereafter, nor are they assigned to classes based on any course subject scores.

近些年不少文献也使用该数据中的随机分班样本进行教育学研究(Gong et al.,2018,2021;Hu,2018; 李长洪和林文炼,2019 等等)。

This study primarily uses CEPS baseline survey data from 2013-2014. School data are often self-selective, i.e., students are assigned to specific classes due to potentially observable or unobservable factors, which can lead to bias in estimating the educational production function. Ideally, in this study, individual student characteristics and classroom characteristics would not affect the class assignment outcomes, i.e., students would be randomly assigned to different classes. Although the Compulsory Education Law of the People's Republic of China strictly requires that junior high schools should not divide their schools into priority and non-priority classes, some schools may not strictly adhere to this rule in practice. {For this reason, the following steps were used to screen schools that may not strictly adhere to the random assignment rule.} First, each principal was asked to respond to two questions about whether 7th graders were randomly or equally assigned to classes upon entry, and whether these students would be reassigned to classes later in secondary school (i.e., in 8th and 9th grades). Schools that reported non-random assignment of students in grade 7 or reassignment in grades 8 and 9 were excluded from this study. Second, classroom teachers were asked to report whether students were assigned based on test scores. Schools whose classroom teachers reported being assigned by performance were also dropped from this study. Finally, the sample for this study includes approximately 10,000 students in grades 7 and 9 from 256 classes in 64 schools.

The key variable of interest is the educational background of peers' parents. We measure it by the average college attainment of classmates’ mothers and fathers separately. To avoid spurious correlation, we follow the approach by Moffitt et al. (2001) to calculate the ‘leave-me-out' average. We use pre-determined variables of students to implement balancing tests for the random assignment and include them as controls in the main specifications. The control variables include age, gender, ethnicity, rural status, local residency status, only child indicator, whether one attended kindergarten, student's age attending primary school, whether one repeated Or skipped in primary school, non-cognitive measures, and parental college education attainment. In Table 1, we provide summary statistics for the aforementioned variables.a

Data are from the Behavioral Risk Factor Surveillance System, a large nationally representative telephone survey of the non-institutionalized adult population administered by the Center for Disease Control and Prevention. Between 2001 and 2005 all states participated. We drop any pregnant women from the analysis as physical activity recommendations are dependent on prior physical fitness. Annual sample sizes range from approximately 112,000 to over 258,000 leading to a combined sample size of over 1 million observations when all four years are used.

概念界定:此前,学术界对流动儿童的概念没有统一、权威的定论。学者周皓将流动儿童的概念界定为6—14周岁且随父母或其他监护人在流入地暂时居住半年以上的儿童少年,又称“流动人口子女”、“进城务工人员子女”等。流动子女与留守子女。相对应。学者范先佐认为流动儿童是指跟随父母移居城市上学的进城务工就业的农民子女,与父母进城务工就业而将子女留在老家的“留守儿童”相对应本研究对农村大龄流动儿童界定有以下两方面限制:1.正在接受初中阶段义务教育的儿童、2.来自农村,且与父母同住。本研究在界定目标人群时,首先依据户籍类型、是否流动两项变量,再根据对农村大龄流动儿童的界定,从整体样本中抽取出2130名拥有农业户籍、跨省及省内流动的流动儿童,其中男性、女性流动儿童分别有1125名、1005名;抽取了7455名拥有城市户籍、没有流动经历且与父母同住的城市儿童,其中男性、女性城市儿童分别有3777名、3678名。(一)不同生活环境的儿童、不同性别流动儿童主观幸福感的差异(二)农村大龄流动儿童主观幸福感的相关影响因素

\subsection{Measurement}

For the non-cognitive measures: CEPS asks students seven questions on personality traits at Grade 6. The first three questions measure persistence, while the rest measures other personality traits. We follow the approach by Zou (2020) and Gong, Lu and Song (2018) to obtain a “persistence” index by averaging the responses of the first three questions and a “other non-cognitive” measure by averaging the responses of the rest four.

被解释变量:本研究衡量学生行为的主要指标是学生体育锻炼时间。体育锻炼是影响青少年健康的重要因素,比如(Y2),甚至会影响接受人力资本的形成。

其中,认知技能是用学生期中考试的三门主课成绩和 CEPS 提 供的标准化认知能力考试成绩来衡量。学校的行政办提供了学生三门核心科目 (语文、数学和英语)的期中考试成绩。学业成绩是学生未来社会经济状况的一个非常重要的预测因素,尤其是初中阶段的学习成绩,直接决定着他们进入高中 和大学的可能性和质量。

认知与非认知技能都是影响学生未来发展的重要因素,比如影响学生接受高等教 育的机会和劳动力市场的结果(Heckman and Rubinstein,2001;Heckman et al, 2006 and 2013)。

解释变量:
本研究的关键解释变量有两个(X1-X5)

父母情感投入维度:情感投入是指在一项活动或者事物中,全身心地投入,并且感情深深地投入其中。这可能是一项艺术活动,如阅读、写作、画画或音乐,也可能是一项爱好,如运动或兴趣爱好。情感投入也可以指在工作或与他人相处方面的投入。情感投入是一种积极的心态,它能帮助人们更加全身心地投入活动,获得更多的乐趣和满足感。指标表现在:对子女以下问题的关心:

教养风格(PS):



其一是学生个人是否接受过学前教育,这是一个虚拟变量(是,赋值为 1);

其二是班级中接受过学前教育的学生的比例。 CPES学生问卷中咨询学生 3 岁以后有没有上过幼儿园/学前班?这里将回答为 “是”的学生赋值为 1;回答为“否”的学生赋值为 0。
【进一步,将这个变量汇总到班级层面,计算出每一个班级中每位学生除了自己之外的班级中接受学前教育的学生占比。】

在具体的操作中,首先生成每个人是否接受学前教育的虚拟变量,然后得到班级中学前 教育的学生占比(除自己之外),此时得到的这两个核心解释变量的均值应该是一致的。


如前所述,已有的各种研究表明,早期人力资本投资对认知和非认知技能在青年期和成年期的成长有极其重要的影响(Almond et al.,2018;Currie and Almond,2011;Cunha and Heckman,2007)。最近的一项研究为本文提供了支持,Cornelissen and Dustmann(2019)利用英国小学独特的入学规则,来识别和 估计学前教育的因果效应。他们发现了学前教育的持久影响,接受过学前教育的 学生通常比没有接受过的学生在智力和情感发展方面表现得更好;随着年龄的增长,学生表现出更高的学习兴趣并且在课堂上表现出较少的破坏性行为(例如破坏课堂纪律或对同学产生的暴力行为)。【进一步,他们还发现处境不利(家庭困难)的男孩从学前教育中受益最大,这可能是由于学前教育为男孩提供了比家中更为健康的环境。】Cornelissen and Dustmann(2019)的研究表明,接受过学前教 育的学生通常拥有更强的能力;那么如果班级中接受学前教育的学生越多,整个 班级的平均人力资本存量就会越高。如果不同班级的平均人力资本存量存在随机 性的差异,这种差异是否会影响不同学校和班级的学生的人力资本发展呢?

parenting style是指家长对于育儿的具体方式和理念。家长可能会采取不同的育儿方式,这些方式可以归纳为四种主要的育儿风格:权威型、宽容型、放任型和严厉型。权威型育儿风格是指家长对孩子实施科学、合理的管教方式,同时也尊重孩子的个性和独立性;宽容型育儿风格是指家长尊重孩子的意愿,允许孩子自由选择和实现自己的目标;放任型育儿风格是指家长缺乏约束和管教,不加干预孩子的行为;严厉型育儿风格是指家长高度控制孩子的行为,采取严格的惩罚措施。不同的育儿风格可能会对孩子产生不同的影响,家长应根据孩子的特点和需求,选择最适合孩子的育儿方式。

控制变量:
个体特征控制变量:性别(女生=1)、民族(汉=1)、同胞数量、户籍类型(农村=1),这些变量属于事前变量

班级特征控制变量:班级规模、班级性别组成、班级的学生家庭经济状况、班主任年龄、性别和班主任职称



\subsection{Statistical analysis}


