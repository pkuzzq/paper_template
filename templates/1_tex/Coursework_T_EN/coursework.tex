%\RequirePackage[no-math]{fontspec} %解决ctex与mathspec冲突
\documentclass[12pt]{article}
\usepackage[a4paper,top=2.5cm,bottom=2.5cm,left=2.5cm,right=2.5cm]{geometry}
\usepackage{amsmath,amsfonts}
\usepackage{longtable}
\usepackage{graphicx}
\usepackage{indentfirst}   %首行缩进宏包
\usepackage[f]{esvect}         %可使用\vv
\usepackage{color}
%\definecolor{ColorName}{rgb}{r,g,b} %这时r/g/b的定义域就在[0-1]。
\definecolor{pkured}{RGB}{126,0,13}%这时R/G/B的定义域就在[0-255]。
\linespread{1.5}
\setlength{\parskip}{0.2\baselineskip plus 0.2ex minus 0.1ex}%段落间的距离

%%%使某些字线重统一%%%%%%%%%%%%%%%%%%%%
\usepackage[full]{textcomp}

\usepackage[sups]{fbb} %
\usepackage{newtxmath} % libertine couled be removed
\usepackage[cal=cm,bb=ams,scr=rsfs]{mathalfa}

%\usepackage[sups]{Baskervaldx} %
%\usepackage[baskervaldx]{newtxmath} % bk mathfont
%\usepackage[cal=cm,bb=ams,scr=rsfs]{mathalfa}

%\usepackage{newtxtext}
%\usepackage{newtxmath}
%\usepackage[cal=cm,bb=ams,scr=rsfs]{mathalfa}

%\usepackage[sb]{libertine}
%\usepackage[libertine,bigdelims]{newtxmath} % 
%\usepackage[cal=cm,bb=ams,scr=rsfs]{mathalfa}

%\usepackage{newpxtext} %
%\usepackage[bigdelims]{newpxmath} % 
%\usepackage[cal=cm,bb=ams,scr=rsfs]{mathalfa}

%\usepackage[scaled=.95]{heuristica} %
%\usepackage[utopia,bigdelims]{newtxmath} %
%\usepackage[cal=cm,bb=ams,scr=rsfs]{mathalfa}

%\usepackage[oldstylenums]{kpfonts}


%\usepackage{ebgaramond}
%\usepackage{newtxmath}
%\usepackage{mathrsfs,amssymb}%when using no fonts

%%与documentclass前一行代码同注同存-------
%\usepackage{mathspec}%在amsmath之后
%\setallmainfonts{Ocean Sans MT Light} %
%\setmainfont[ItalicFont={Ocean Sans MT Light Italic}, BoldFont={Ocean Sans MT}]{Ocean Sans MT Light} %
%\usepackage{mathrsfs,amssymb}%when using no fonts

%%%adobe-garamond字体数学符号配置方案%%%%%%%
%\usepackage[adobe-garamond]{mathdesign}
%\usepackage[T1]{fontenc}
%%%%math(bf,it,cal,scr)------------------
%\makeatletter
%%%%urw-garamond时,把mdpgd换为mdugm------
%%%%\let\mathbf\undefined%adobe时得注释掉
%\DeclareMathAlphabet{\mathbf}{OT1}{mdpgd}{b}{n}
%%%%\let\mathit\undefined%adobe时得注释掉
%\DeclareMathAlphabet{\mathit}{OT1}{mdpgd}{m}{it}
%%---------------------------------------
%\DeclareSymbolFont{xsymbols}{OMS}{cmsy}{m}{n}
%\DeclareSymbolFontAlphabet{\mathcal}{xsymbols} 
%\DeclareSymbolFont{xmathscript}{U}{rsfs}{m}{n}
%\DeclareSymbolFontAlphabet{\mathscr}{xmathscript}
%\makeatother
%%%%%%%%%%%%%%%%%%%%%%%%%%%%%%%%%%%%%%%%

%%%define new fonts%%%%%%%%%%%%%%%%%%%%%
%\usepackage{fontspec}
%\newcommand{\eclight}{\fontspec{Ethnocentric-Light}}
%%%%%%%%%%%%%%%%%%%%%%%%%%%%%%%%%%%%%%%%

%%%节标题加色%%%%%%%%%%%%%%%%%%%%%%%%%%%%%
\usepackage{titlesec}
\titleformat{\section}
{\Large\bfseries}
{}
{0.5em}
{\color{pkured}}
%%%%%%%%%%%%%%%%%%%%%%%%%%%%%%%%%%%%%%%%

%%%list编号相关%%%%%%%%%%%%%%%%%%%%%%%%%%
%%%enumerate item编号-------------------
\renewcommand{\labelenumi}{(\theenumi)} %\labelenumi (1级e-item)
\renewcommand{\labelenumii}{({\it\theenumii})}%\labelenumii (2级e-item)
%%%编号左边距----------------------------
\setlength{\leftmargini}{2em}%一级编号顶格
\setlength{\leftmarginii}{2em}%二级编号缩进1.5格
\setlength{\leftmarginiii}{2em}%三级编号缩进1.1格
\renewcommand{\labelitemi}{$\bullet$}      %修改itemize二级编号
\renewcommand{\labelitemii}{$\circ$}      %修改itemize二级编号
\renewcommand{\labelitemiii}{$\diamond$}      %修改itemize二级编号
%%%设置item间距-------------------------
\usepackage{enumitem}
\setenumerate[1,2,3]{itemsep=0pt,partopsep=0pt,parsep=0pt,topsep=0pt}
\setitemize[1,2,3]{itemsep=0pt,partopsep=0pt,parsep=0pt,topsep=0pt}
\setdescription{itemsep=0pt,partopsep=0pt,parsep=0pt,topsep=0pt}

%%%画线%%%%%%%%%%%%%%%%%%%%%%%%%%%%%%%%%
\usepackage{xhfill}
%%%%%%%%%%%%%%%%%%%%%%%%%%%%%%%%%%%%%%%%

%%%解决amsthm中openbox与xhill冲突问题%%%%%%
\let\openbox\relax
%%%设置定理环境%%%%%%%%%%%%%%%%%%%%%%%%%%%
\usepackage{amsthm}
\newtheorem{theorem}{Theorem}
\newtheorem*{theorem*}{Theorem}
\newtheorem{corollary}[theorem]{Corollary}
\newtheorem{lemma}[theorem]{Lemma}
\newtheorem{proposition}[theorem]{Proposition}
\newtheorem*{claim*}{Claim}
\newtheorem{exercise}[theorem]{Exercise}
\theoremstyle{definition}
\newtheorem{definition}[theorem]{Definition}
\newtheorem{remark}[theorem]{Remark}
\newtheorem{notation}[theorem]{Notation}
%%%习题编号改为x.%%%%%%%%%%%%%%%%%%%%%%%%%
\renewcommand\thetheorem{1.\arabic{theorem}}%roman,Roman,alph,Alph
%%%%%%%%%%%%%%%%%%%%%%%%%%%%%%%%%%%%%%%%

%%%设置页眉页脚%%%%%%%%%%%%%%%%%%%%%%%%%%%
\usepackage{fancyhdr}
\renewcommand{\headrulewidth}{0pt}
%%%%%%%%%%%%%%%%%%%%%%%%%%%%%%%%%%%%%%%%


\title{\bfseries COURSEWORK 1%Weekly Coursework
}
\author{G. Hilbert%Assitant Professor Conden Chao
%\\ School of Philosophy, Renmin University of China
}
\date{\bf Deadline: 4/4/2018}
\begin{document}
\noindent\includegraphics[height=1cm]{pku.png}\hfill{\raisebox{.38\baselineskip}{\textcolor{pkured}{\bfseries\scshape Mathematical Logic in 2018 Fall}}}
\vskip -22pt
{\let\newpage\relax\maketitle}

%%%设置页码格式%%%%%%%%%%%%%%%%%%%%%%%%%%%
\thispagestyle{fancy}
\cfoot{\small\thepage$/$\pageref{CC001}}
%%%%%%%%%%%%%%%%%%%%%%%%%%%%%%%%%%%%%%%%

%%%设置页码格式%%%%%%%%%%%%%%%%%%%%%%%%%%%
\pagestyle{fancy}
\cfoot{\small\thepage~\textit{of}~\pageref{CC001}}
%%%%%%%%%%%%%%%%%%%%%%%%%%%%%%%%%%%%%%%%

\noindent\xrfill{1pt}[pkured]
~\textcolor{pkured}{\bfseries INSTRUCTIONS}~
\xrfill{1pt}[pkured]  
\textcolor{pkured}{
\begin{itemize}
\item Write your name and student ID clearly on the papers;
\item Submit your coursework to the assistant before the deadline;
\item Copy is forbidden, otherwise the score for the coursework would be 0 this time;
\item Typewriting with \LaTeX{} is encouraged that 1 more score will be awarded this time.
\end{itemize}
}
\noindent\xrfill{1pt}[pkured] 
\par\vskip 12pt 

\section*{Warm-up}
\begin{exercise}
The	same thing is $f(x)=g(x)=\mathbf{S}=\mathrm{Ox}$.
\end{exercise}

\begin{exercise}
	The	same thing is 
	\[\prod_{x=y=0}^{6,9}f(x,y)=g(x,y)=2x^2+y+2018.\]
\end{exercise}

\section*{Medium}
\begin{definition}
	Suppose $M$ is a non-empty set and $P\subseteq M$.
	\begin{enumerate}
		\item $X$ is definable over $(M,\in)$ from parameters in $P$ if there is an $\mathscr{L}_\in$ formula $\varphi(x,\vv{y})$ and some $\vv{b}$ from $P$ such that
		\[
		X=\{a\in M\mid (M,\in)\vDash \varphi(x,\vv{y})[a,\vv{b}]\}.
		\]
		\item $D(M)=\{X\subseteq M\mid X\text{ is definable over }(M,\in)\text{ from parameters in }M\}$.
		\item In particular, $D(\varnothing)=\{\varnothing\}$.
	\end{enumerate}
\end{definition}

\begin{definition}[$\mathsf{ZF}$-$\mathsf{P}$, G\"odel]
	By transfinite recursion on $\alpha\in\mathsf{ON}$, 
	define $L_\alpha$ by
	\begin{equation*}
	\begin{array}{rcll}
	L_\alpha&=&\varnothing,&\alpha=0,\\
	L_{\alpha}&=&D(L_\delta),&\alpha=\delta+1,\\
	L_\alpha&=&\bigcup_{\delta<\alpha}L_\delta,&\alpha\text{ is a limit}.
	\end{array}
	\end{equation*}
	And set $L=\bigcup_{\alpha\in\mathsf{ON}}L_\alpha$, and we call $L$ the constructible universe.
\end{definition}

\section*{Advanced}
\begin{definition}
	$X$ is transitive if any element of $X$ is a subset (or subclass) of $X$.
\end{definition}

\begin{lemma}
	For any ordinals $\zeta,\xi$,
	\begin{enumerate}
		\item $L_\zeta\subseteq V_\zeta$.
		\item $L_\zeta$ is transitive.
		\item if $\zeta<\xi$, then $L_\zeta\subseteq L_\xi$.
		\item if $\zeta<\xi$, then $L_\zeta\in L_\xi$.
		\item $L_\zeta\cap\mathsf{ON}=\zeta$.
		\item $L$ is transitive.
		\item $L$ is a proper class.
	\end{enumerate}
\end{lemma}
$\mathbf{A,B,C,D,S,O,P,u,v,w,x,y,z}$
\clearpage

$\mathsf{EA}$, the theory of so-called elementary arithmetic,
which is what you get by taking $\mathsf{Q}$ and adding $\Delta_0$-induction plus the axiom which says that $2^n$ is a total function.



\label{CC001}
%{\setlength{\LTleft}{0pt} \setlength{\LTright}{0pt} \small
%%表格与页面左右边缘之间的矩离均为0%%%%%%%%
%\begin{longtable}{@{\extracolsep{\fill}}r|l|r|l|r|l}
%%@{\extracolsep{\fill}}设置使得后面所有列间距可以伸展到预定义的表格宽度。
%\hline
%Number&19870820&Title&Mathematical Logic&Attributes& Professional Required Course\\\hline
%Semester&2018 Fall&Level&Undergraduate&College&School of Philosophy\\\hline
%Teacher&Asst. Prof. Conden Chao&Phone&18001166719&Email&zhaoxy00@126.com\\\hline
%Assistant&&Phone&&Email&\\\hline
%\end{longtable}
%}
\end{document}