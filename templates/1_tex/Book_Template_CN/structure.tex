%%%设置中英文混排时中文文字体%%%%%%%%%%%%%%%%
\setCJKmainfont[ItalicFont={FZKai-Z03S}, BoldFont={FZHei-B01S}]{FZShuSong-Z01S} %
\setCJKsansfont{FZHei-B01S} %
\setCJKmonofont{FZFangSong-Z02S} %
%%%定义新字体%%%%%%%%%%%%%%%%%%%%%%%%%%
\setCJKfamilyfont{kaiti}{FZKai-Z03S} 			%使用楷体
\newcommand{\kai}{\CJKfamily{kaiti}}
%%%%%%%%%%%%%%%%%%%%%%%%%%%%%%%%%%%

%%%目录格式设置%%%%%%%%%%%%%%%%%%%%%%%%%
\renewcommand{\contentsname}{目\hspace{1em}录}	%把目录中间加空格
%%%%%%%%%%%%%%%%%%%%%%%%%%%%%%%%%%%

%%%中文章节格式设置%%%%%%%%%%%%%%%%%%%%%%%
\usepackage{titlesec}
%章格式设置---------------------------------------------------
\CTEXsetup[name={第~ ,~章},number={\arabic{chapter}}]{chapter} 
\titleformat{\chapter}{}{\centering\bfseries\huge 第~{\bfseries\thechapter}~章}{1em}{\centering\huge\bfseries}
%%%以下名称ctex宏包默认,故不需要改%%%%%%%%
%\renewcommand\refname{参考文献} % article类文档
%\renewcommand\bibname{参考文献} % report/book类文档
%\renewcommand{\figurename}{图}
%\renewcommand{\tablename}{表}
%\renewcommand{\appendixname}{附录}
%\renewcommand{\algorithm}{算法}

%%%章结尾页若为空白页则注“此页故意留白”%%%%%%%%%%%
\makeatletter
\renewcommand{\cleardoublepage}{
\clearpage\ifodd\c@page\else
\hbox{}
\vspace*{\fill}
\thispagestyle{empty}
\begin{center}\nbvspace[1]\textit{\normalsize 此页有意留白}\nbvspace[4]\end{center}
\newpage
\fi}
\makeatother
%%%%%%%%%%%%%%%%%%%%%%%%%%%%%%%%%%




%%%设置页眉页脚%%%%%%%%%%%%%%%%%%%%%%%%
\usepackage{fancyhdr}%调用宏包
%设置新的plain格式-------------------------------------------
\fancypagestyle{plain}{
	\fancyhf{}
	\fancyhead{}
	\fancyfoot{}
	\renewcommand{\headrulewidth}{0pt}}
\renewcommand{\headrulewidth}{0pt} % 页眉与正文之间的水平线粗细,默认0.4pt
%\renewcommand{\footrulewidth}{1pt}
%\setlength{\topskip}{0mm}    %页眉与正文间距离
%%%%%%%%%%%%%%%%%%%%%%%%%%%%%%%%%%%

%%%设置脚注%%%%%%%%%%%%%%%%%%%%%%%%%%%
\usepackage[perpage,stable,bottom]{footmisc}
\setlength{\footnotesep}{0.6\baselineskip}% 脚注与脚注之间行距
\setlength{\skip\footins}{1\baselineskip plus 0.6ex minus 0.3ex}% 脚注与正文间距
%脚注编号带圈儿----------------------------------------------
\usepackage{pifont}%调用带圈的数字
\renewcommand\thefootnote{\ding{\numexpr171+\value{footnote}}\,}%脚注编号设成带圈数字rm
%正文中脚注编号上标,脚注中编号非上标----------------------
\makeatletter
\let\old@makefntext\@makefntext
\long\def\@makefntext#1{%
	%\quad %10号默认字体下不加这个
	\def\@makefnmark{\hbox{\footnotesize\normalfont\@thefnmark}}%
	\old@makefntext{#1}}
\makeatother
%%%%%%%%%%%%%%%%%%%%%%%%%%%%%%%%%

%%%设置一系列间距%%%%%%%%%%%%%%%%%%%%%%
\usepackage{indentfirst}   %首行缩进宏包
\linespread{1.4}%
\setlength{\footskip}{1.4\baselineskip}%页码与正文距离
\setlength{\parskip}{0.3\baselineskip plus 0.2ex minus 0.1ex}%段落间的距离
%%%设置item间距%%%%%%%%%%%%%%%%%%%%%
\usepackage{enumitem}
\setenumerate[1,2,3]{itemsep=0pt,partopsep=0.3\baselineskip plus 0.2ex minus 0.1ex,parsep=0pt,topsep=0pt}
\setitemize[1,2,3]{itemsep=0pt,partopsep=0.3\baselineskip plus 0.2ex minus 0.1ex,parsep=0pt,topsep=0pt}
\setdescription{itemsep=0pt,partopsep=0.3\baselineskip plus 0.2ex minus 0.1ex,parsep=0pt,topsep=0pt}

\renewcommand{\arraycolsep}{1.5pt}%array环境中对齐项两侧间距大小
%%%%%%%%%%%%%%%%%%%%%%%%%%%%%%%%%%%

%%%修改单独成行公式与文本之间间距%%%%%%%%%%%%%%
\usepackage{etoolbox}
\makeatletter
\appto\normalsize{%
	\abovedisplayskip 6 \p@ \@plus1\p@ \@minus3\p@
	\abovedisplayshortskip \z@ \@plus3\p@
	\belowdisplayshortskip 3\p@ \@plus1\p@ \@minus2\p@
	\belowdisplayskip \abovedisplayskip
}
\makeatother
%%%%%%%%%%%%%%%%%%%%%%%%%%%%%%%%%

%%%编号相关%%%%%%%%%%%%%%%%%%%%%%%%%
\renewcommand{\labelenumi}{(\theenumi)} %\labelenumi (1级e-item)
\renewcommand{\labelenumii}{({\it\theenumii})}%\labelenumii (2级e-item)
%%%设置item缩进值%%%%%%%%%%%%%%%%%%%%%%
%需要enumitem宏包---------------------
\setlist[enumerate]{leftmargin=1.7em}
\setlist[itemize]{leftmargin=1.01em}
%%%定义随字号变化的阴阳符号%%%%%%%%%%%%%
%需要tikz宏包------------------------
\usepackage{scalerel}
\newcommand{\customyinyang}{%
	\scalerel*{%
		\tikz[anchor=base, baseline]{%
			\draw[line width = 0.05em] (0,0.26em) circle (0.17em);
			\path[fill=black] (90:0.43em) arc (90:-90:0.085em)
			(0,0.43em)    arc (90:270:0.17em)
			(0,0.09em) arc (-90:-270:0.085em);
		}%
	}{\bullet}%
}
\renewcommand{\labelitemi}{$\bullet$}%修改itemize二级编号
\renewcommand{\labelitemii}{$\customyinyang$}%修改itemize二级编号
\renewcommand{\labelitemiii}{$\circ$}%修改itemize二级编号
%%%%%%%%%%%%%%%%%%%%%%%%%%%%%%%%%%%%%

%%%图表格式设置%%%%%%%%%%%%%%%%%%%%%%%%%%%
%设置图编号随章---------------------------------------------------
\renewcommand\thefigure{${\thechapter.\arabic{figure}}$\,}%默认的图片编号本如此,只是数字后的冒号挨太紧
%设置表编号格式---------------------------------------------------
\renewcommand\thetable{${\thechapter.\arabic{table}}$\,}%默认的图片编号本如此,只是数字后的冒号挨太紧
%解决表格标题距内容过近------------------------------------------
\usepackage{caption}
%表格线加粗-------------------------------------------------------
\usepackage{booktabs}
%设置图表标题与文本之间的间距------------------------------------
\setlength{\abovecaptionskip}{6pt plus 1pt minus 3pt}
\setlength{\belowcaptionskip}{0pt}
%%%%%%%%%%%%%%%%%%%%%%%%%%%%%%%%%%%%%

%%%参考文献总设置%%%%%%%%%%%%%%%%%%%%%%%%%%
\usepackage[backend = biber, style = caspervector-ay, utf8,
maxbibnames=999,maxcitenames=999,%引用和列表的文献作者全显示
sortcites=false,%引用显示按照输入顺序
sorting = cenyt%列表显示按照cenyt
]{biblatex}
%%%参考文献格式修改%%%%%%%%%%%%%%%%%%%%%%%%%
\makeatletter
%列表文献标题设置-------------------------------------------------
\renewcommand*{\bbx@cepunct}[2]{#2}
\DeclareFieldFormat*{title}{\bbx@cetext{《{#1}》}{\mkbibemph{#1}}}
\DeclareFieldFormat*{booktitle}{\bbx@cetext{《{#1}》}{\mkbibemph{#1}}}
\DeclareFieldFormat*{journaltitle}{\bbx@cetext{《{#1}》}{\mkbibemph{#1}}}
\DeclareFieldFormat[inbook, inproceedings, incollection, article]
{title}{\bbx@cetext{《{#1}》}{{#1}}}%\mkbibquote可使此类title加双引号----
\DeclareFieldFormat*{volume}{{#1}}%卷号不再斜体;\mkbibbold可加粗
%\DeclareFieldFormat*{type}{\mkbibbrackets{\mkbibemph{#1}}}%文献类型C,M,J斜体
%列表中2个以上作者时,中文后2个加‘和’,英文后2个加‘and’
\renewcommand*{\finallistdelim}{\bbx@cetext{{和}}{\addspace\bibstring{and}\space}}
\renewcommand*{\finalnamedelim}{\bbx@cetext{{和}}{\addspace\bibstring{and}\space}}
%去掉英文文献两个以上作者时最后两个作者之间加的and
%\renewcommand*{\finallistdelim}{\bbx@cecomma}
%\renewcommand*{\finalnamedelim}{\bbx@cecomma}%国标
%列表中姓名显示姓在前---------------------------------------------
\DeclareNameAlias{author}{last-first}
\DeclareNameAlias{editor}{last-first}
\DeclareNameAlias{translator}{last-first}
%%引用显示和列表显示,作者和年份之间加英文逗号和英文空格
%\DeclareDelimFormat*{nameyeardelim}{\addcomma\space}
%引用显示加括号---------------------------------------------------
\renewcommand{\cite}{\Parencite}
%引用显示括号为中括号---------------------------------------------
\AtEveryCite{%
	\let\bibopenparen=\bibopenbracket%
	\let\bibcloseparen=\bibclosebracket}
%引用显示顿号改为英文逗号+英文空格------------------------------
\renewcommand{\bbx@cnbcomma}{\addcomma\space}
\makeatother
%去掉列表中年份的括号并在年份前加英文逗号和空格---
\usepackage{xpatch}
\xpatchbibmacro{date+extradate}{%
	\printtext[parens]%
}{%
	\setunit*{\addcomma\space}%
	\printtext%
}{}{}
%增加列表项间距---------------------------------------------------
\setlength{\bibitemsep}{0.3\baselineskip plus 0.2ex minus 0.1ex}
%导入数据库-------------------------------------------------------
\addbibresource{myref.bib}
%%%%%%%%%%%%%%%%%%%%%%%%%%%%%%%%%%%%%

%%%设置索引%%%%%%%%%%%%%%%%%%%%%%%%%%%%%
\usepackage{imakeidx}%添加索引
\makeindex
%%%添加超链接%%%%%%%%%%%%%%%%%%%%%%%%%%%%
\usepackage[colorlinks=false,breaklinks=true,
bookmarks=true,bookmarksopen=true]{hyperref}
%该宏包必须在imakeidx之后,不然index页码无超链接
%%%符号索引%%%%%%%%%%%%%%%%%%%%%%%
\makeindex[name=symbol,title=符号索引,options=-s mystyle1.ist]
%%%译名索引%%%%%%%%%%%%%%%%%%%%%%%
\makeindex[name=name,title=名称索引,options=-s mystyle2.ist]
%%%术语索引%%%%%%%%%%%%%%%%%%%%%%%%%%
\renewcommand{\indexname}{术语索引}%%%三种索引都分两栏%%%%%%%%%%%%%%%%%%%
\usepackage[columns=2,totoc=true,rule=0pt,indentunit=1em,columnsep=20pt]{idxlayout}
%分栏,显示在目录中,栏间加线,子项缩进,栏间距离
%%%符号索引格式%%%%%%%%%%%%%%%%%%%%%
\usepackage{filecontents}
\begin{filecontents}{mystyle1.ist}
	preamble "
	\\begin{theindex}
	\\providecommand*\\indexgroup[1]{\\indexspace
		\\item \\textbf{#1}}
	"
	postamble "\n\n\\end{theindex}\n"
	group_skip "  %\n  \\indexspace\n  %\n"
	headings_flag 1           %如果索引项前加字母,把0换成1
	heading_prefix "  %\n  \\indexgroup{"
	heading_suffix "}\n  %\n"
%suffix_2p "\\nohyperpage{-f.}"
%suffix_3p "\\nohyperpage{ff.}"
delim_0 "\\dotfill"           %索引项和页码之间用点填充用dotfill;不用加点则用hfill;也可用qquad
delim_1 "\\dotfill"           %索引项和页码之间用点填充用dotfill;不用加点则用hfill;也可用qquad
\end{filecontents}
%%%译名索引格式%%%%%%%%%%%%%%%%%%%%%%%%%%
\usepackage{filecontents}
\begin{filecontents}{mystyle2.ist}
preamble "
\\begin{theindex}
\\providecommand*\\indexgroup[1]{\\indexspace
	\\item \\textbf{#1}}
"
postamble "\n\n\\end{theindex}\n"
group_skip "  %\n  \\indexspace\n  %\n"
headings_flag 1           %如果索引项前加字母,把0换成1
heading_prefix "  %\n  \\indexgroup{"
heading_suffix "}\n  %\n"
%suffix_2p "\\nohyperpage{-f.}"
%suffix_3p "\\nohyperpage{ff.}"
delim_0 "\\dotfill"           %索引项和页码之间用点填充用dotfill;不用加点则用hfill;也可用qquad
delim_1 "\\dotfill"           %索引项和页码之间用点填充用dotfill;不用加点则用hfill;也可用qquad
\end{filecontents}
%%%术语索引格式%%%%%%%%%%%%%%%%%%%%%%%%%
\begin{filecontents}{\jobname.mst}
preamble "
\\begin{theindex}
\\providecommand*\\indexgroup[1]{\\indexspace
\\item \\textbf{#1}}
"
postamble "\n\n\\end{theindex}\n"
group_skip "  %\n  \\indexspace\n  %\n"
headings_flag 1           %如果索引项前加字母,把0换成1
heading_prefix "  %\n  \\indexgroup{"
heading_suffix "}\n  %\n"
%suffix_2p "\\nohyperpage{-f.}"
%suffix_3p "\\nohyperpage{ff.}"
delim_0 "\\dotfill"           %索引项和页码之间用点填充用dotfill;不用加点则用hfill;也可用qquad
delim_1 "\\dotfill"           %索引项和页码之间用点填充用dotfill;不用加点则用hfill;也可用qquad
%delim_2 "\\hfill"
\end{filecontents}
%%%%%%%%%%%%%%%%%%%%%%%%%%%%%%%%%%%%

%%%设置定理环境%%%%%%%%%%%%%%%%%%%%%%%%
\usepackage{amsthm}
%中文定理与定义格式-----------------------------------------
\makeatletter
\newtheoremstyle{mythm}%名字
{0.3\baselineskip plus 0.1ex minus 0.1ex}%上方的空行
{0.3\baselineskip plus 0.1ex minus 0.1ex }%下方的空行
{\kai}%主体内容,可在此修改内容字体
{}%首行缩进
{\bfseries }%定理头部字体
{}%定理头部后的标点
{0.5em}%定理头部后的空格
{\thmname{#1}\thmnumber{#2}\thmnote{(#3)\!\!}.}%定理头部的说明
\makeatother
\makeatletter
\newtheoremstyle{mydefn}%名字
{0.3\baselineskip plus 0.2ex minus 0.1ex}%上方的空行
{0.3\baselineskip plus 0.2ex minus 0.1ex}%下方的空行
{}%主体内容,可在此修改内容字体
{}%首行缩进
{\bfseries }%定理头部字体
{}%定理头部后的标点
{0.5em}%定理头部后的空格
{\thmname{#1}\thmnumber{#2}\thmnote{(#3)\!\!}.}%定理头部的说明
\makeatother
%1和2换过来之后,}之后加~,否则交换无效%%
%设置定理-----------------------------------------------------
\theoremstyle{mythm}
\newtheorem{theorem}{定理}[section]
\newtheorem{lemma}[theorem]{引理}
\newtheorem{corollary}[theorem]{推论}
\newtheorem{claim}[theorem]{断言}
\theoremstyle{mydefn}
\newtheorem{definition}[theorem]{定义}
\newtheorem{remark}[theorem]{评注}
\newtheorem{example}[theorem]{示例}
\newtheorem{exercise}[theorem]{习题}
\newtheorem{convention}[theorem]{约定}
\newtheorem{notation}[theorem]{记号}
%\newtheorem{thesis}[theorem]{论题}
%\newtheorem{conjecture}[theorem]{猜想}
%\newtheorem{problem}[theorem]{问题}
%\newtheorem{axiom}[theorem]{公理}
%%%%%%%%%%%%%%%%%%%%%%%%%%%%%%%%%%%

%%%中文证明环境%%%%%%%%%%%%%%%%%%%%%%%%%
\makeatletter
\renewenvironment{proof}[1][\proofname]{\par
\pushQED{\qed}%
\normalfont \topsep0\p@\@plus0\p@\relax   %证明环境前后行距
\trivlist 
\item[\hskip\labelsep % 首行缩进则加\indent 
\bfseries 
#1\@addpunct{:}]\ignorespaces  %:为证明后的冒号
}{%
\popQED\endtrivlist\@endpefalse
}
\renewcommand{\proofname}{证明} %Proof的名称换为证明
\makeatother
%证明结束符修改-----------------------------------------------
\renewcommand{\qedsymbol}{\ensuremath\boxtimes}
%%%%%%%%%%%%%%%%%%%%%%%%%%%%%%%%%%%